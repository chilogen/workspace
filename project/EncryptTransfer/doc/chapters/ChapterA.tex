% !Mode:: "TeX:UTF-8"

\chapter{绪论}
\label{Chapter::one}
\section{文件加密传输需求分析}

\section{对称加密与非对称加密对比}
维基百科上对 对称加密的叫法直译为对称密钥加密。顾名思义,对称加密是一种使用相同密钥进行加密解密的一系列算法。它通过同一个密钥对原文进行加密,然后同时将密钥和密文传送给接收方。通俗点说,就是要有一个密码本,比如著名的AES,过时的DES,还有古老的凯撒加密等,都属于对称加密。

非对称加密,又称公钥加密它与对称加密的最大不同在与分别使用不同的密钥进行加密与解密。常见的公钥加密算法有RSA、Rabin、椭圆曲线加密算法等。使用最广泛的是RSA算法。

对称密钥的最大的一个问题是密钥有可能会在传输过程中被窃取,容易遭受中间人攻击,但好处是加密速度快。非对称密钥则可以将公钥告知任何人,任何人都可以使用该公钥进行加密,但只有私钥拥有者才能对其进行解密,但不足之处是加密解密开销比对称加密大得多。

因此,在实际应用中,一般都是二者取其长处。使用对称加密算法对大块数据进行加密,然后使用非对称加密算法加密对称加密使用的密钥,在保证了安全的同时,也兼顾了速度。

\section{本次综合课程设计需要解决的问题与使用的技术}
课题要求实现客户端和服务端两个程序,使用对称\textbackslash 非对称加密算法实现它们之间的安全通信。在加密方面,我选择了AES结合RSA的方法。同时为了实现完整的C/S功能,还使用了如下一些技术:
\begin{enumerate}
\item 简单网络协议

在\ref{section::ENPTP}介绍了一个简单的私有应用层协议
\item AES加密

使用AES对数据块进行加密,在\ref{section::AES}详细介绍。
\item RSA
使用RSA对AES密钥进行加密,在\ref{section::RSA}详细介绍。

\item POSIX线程

在服务器端使用多线程实现了并发连接,同时使用最简单的互斥量完成线程同步。
\item 网络套接字编程

网络通信自然少不了套接字编程,为了实现简单,我采用了面向连接的可靠数据传输协议TCP作为运输层协议
\item CMake管理

为了使项目体现出模块化和快速构建,采用CMake进行项目管理
\end{enumerate}

\section{本论文的结构安排}
本文的章节结构安排如下:

在第二章首先介绍使用的网络协议,包括运输层的TCP协议和应用层的自定义协议\footnote{之后简称ENPTP协议}。在第三章将会对使用到的AES,和RSA加密详细介绍。第四章对Linux下的POSIX线程编程进行说明。第五章则是根据前三章内容构建程序的过程方法和其中需要注意的问题,最后一章是全文总结。
